% ------------------------------------------------
\StartSection{基本介紹 Introduction}{chapter:how-to:use:intro}
% ------------------------------------------------

如果要使用本模版來寫你的論文, 那你要先拿到3個東西:

% ------------------------------------------------
\StartSubSection{本模版的檔案}

本模版的source code (.tex檔, 即是Mircosoft Word的.doc檔) 已經完整的放在GitHub上\RefBib{web:this-project:github} (Fig \RefTo{fig:how-to:use:intro:github}).

\InsertFigure
  [scale=0.18,
    caption={Template on GitHub},
    label={fig:how-to:use:intro:github}]
  {./example/how-to/use/intro/pic/github.png}

可以使用右方的"Download ZIP"來下載最新版的本模版(Fig \RefTo{fig:how-to:use:intro:github-download}).
\InsertFigure
  [scale=0.4,
    caption={Download Icon},
    label={fig:how-to:use:intro:github-download}]
  {./example/how-to/use/intro/pic/github-download.png}

% ------------------------------------------------
\StartSubSection{編寫用Editor}

用來編寫你的論文用的Editor

因為要寫Word的話, 就必須使用Mircosoft Office才可以寫. 但如果是寫LaTex的話, 就算只是記事本都可以編寫, 所以在這邊你可以去使用你喜好的Editor.

但注意的是, 盡量切勿使用一些預設不是針對UTF-8的Editor, 如 Windows內建的記事本(notepad). 因為你在過程中應該都會寫出或留下一些中文, 這時候如果那個Editor自動存成其他編碼(notepad預設會存成ANSI), 就可能會潛存的錯誤存在.

所以使用一些有名的Editor可能會比較保障這問題, 如Notepad++, Gedit等 (當然你都可以使用你熟悉的).

% ------------------------------------------------

\StartSubSection{產生用的工具}

用來讀取LaTex來產生你的論文用的工具

以Mircosoft Word來講是Mircosoft Office, 而本模版則會介紹使用Texmaker + MiKTeX來處理, 請看第\RefPage{chapter:how-to:use:generate}頁的`產生論文 Generate Thesis'.

% ------------------------------------------------
